\documentclass{scrartcl}
\usepackage{graphicx}
\usepackage[update,prepend]{epstopdf}
\usepackage{floatrow}
\usepackage{glossaries}
\usepackage{ifpdf}
\newglossaryentry{A}
{
  name=A,
  description={First letter}
}
\makeglossaries

\begin{document}
\glsaddall
\begin{figure}
\includegraphics[width=1cm]{graphics/gif.eps}
\caption{GIF image}
\end{figure}
\begin{figure}
\includegraphics[width=1cm]{graphics/jpg.eps}
\caption{JPG image}
\end{figure}
\begin{figure}
\includegraphics[width=1cm]{graphics/png.eps}
\caption{PNG image}
\end{figure}
\begin{figure}
\includegraphics[width=1cm]{graphics/svg.eps}
\caption{SVG image}
\end{figure}
\begin{figure}
\includegraphics{graphics/ai.eps}
\caption{graphics/ai.eps}
\end{figure}
\begin{figure}
\includegraphics[width=5cm]{graphics/asy.eps}
\caption{Asymtote image}
\end{figure}
%
\begin{figure}
\ifpdf
	\includegraphics[width=5cm]{graphics/asy_pdf.pdf}
	\caption{Asymtote image directly converted to PDF}
\else
	\caption{WARNING! Cannot test Asymtote to PDF as latex was used to compile!}
\fi
\end{figure}
%
\begin{figure}
\includegraphics{graphics/dia.eps}
\caption{graphics/dia.eps}
\end{figure}
\begin{figure}
\includegraphics{graphics/fig.eps}
\caption{FIG image}
\end{figure}
\begin{figure}
\includegraphics[width=5cm]{graphics/gnuplot.eps}
\caption{Gnuplot image}
\end{figure}
\clearpage
\printglossaries
\end{document}
